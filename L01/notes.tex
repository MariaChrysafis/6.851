%
\documentclass[11pt]{scrartcl}
\usepackage[sexy]{evan}
\title{6.851}
\date{\today}
\begin{document}
\maketitle
\tableofcontents
\section{Week 1}
\subsection{Topics}
\begin{itemize}
	\item \textbf{time travel}: temporal data structures; manipulating time
	\item \textbf{geometry}: higher dimensional data
	\item \textbf{dynamic optimality}: is there one best binary search tree?
	\item \textbf{memory hierarchy}: a way to model more realistic computers which have cache, more cache, main memory, and all these different levels
	\item \textbf{hashing}: most used data structure in CS
	\item \textbf{integers}: when you know that your data structure are integers and not just arbitrary black boxes that you can compare, you could do a lot better with integers sometimes. 
	\item \textbf{dynamic graphs}: graphs where edges are being added and maybe deleted (e.g. social network where people are friending and defriending)
	\item \textbf{strings}: you have a piece of text (e.g. entire world wide web), and you want to search for a substring efficiently
	\item \textbf{succint data structure}: reducing things to bare minimum of space 
\end{itemize}
\subsection{Model of Computation}
Model of computation that you're working with matter. Models matter. Throughout today, we'll be using a pointer machine model. 

You have a bunch of nodes with some constant number of fields in them (e.g. structs in C). Each of the fields could be a pointer to another node, or it could have some data in it (e.g. integers). You can have pointers to yourself. 

Operations that are allowed:
\begin{itemize}
	\item \textbf{Create a new node}: $x = \text{new node}$.
	\item \textbf{Look at fields}: $x = \text{y.field}$.
	\item \textbf{Set Fields}: $\text{x.field} = y$.
	\item \textbf{Perform Operations on Fields}: $x = y + z$.
	\item \textbf{Destroy Nodes}
\end{itemize}
There is a $\textbf{root}$ node as well. 
\subsection{Temporal Data Structure}
Two types of persistence: 
\begin{itemize}
	\item \textbf{Persistence}: branching universe time travel model (if you make a change in the past, you make a new universe). 
		\begin{itemize}
			\item You never destroy old universes.
			\item Keep all versions of data structure
			\item Data structure operations relative to specified version
			\item An update makes and returns  a new version
		\end{itemize}
	\item \textbf{Retroactivity}: go back, make a change, and then you can return back to the present to see what happened.
\end{itemize}
Four levels of persistence:
\begin{itemize}
	\item \textbf{Partial Persistence}
		\begin{itemize}
			\item You are only allowed to update the latest version.
			\item The versions are linearly ordered.
		\end{itemize}
	\item \textbf{Full Persistence}
	\begin{itemize}
		\item Update any version 
		\item The versions form a tree.
	\end{itemize}
\item \textbf{Confluent Persistence}
	\begin{itemize}
		\item Combine two versions to create a new version
		\item Versions form a DAG (Directed Acyclic Graph)
	\end{itemize}
\item \textbf{Functional Persistence}
	\begin{itemize}
		\item Never modify nodes
		\item Only create new nodes
		\item Version of DS represented by pointer
	\end{itemize}
\end{itemize}
\subsection{Partial Persistence}
\begin{claim}
Any pointer machine DS with a  constant number of pointers to any node (constant in-degree) can be transformed into another data structure that is partially persistent with $O(1)$ amortized factor overhead. Constant amount of amortized space per change in the data structure.
\end{claim}
\begin{proof}
	Let our pointer machine have $\le p = O(1)$ pointers to any node for some constant $p$.
Sketch:
	\begin{itemize}
		\item Store all of the back pointers only for the latest version of the DS. 
		\item Store modifications to fields of DS ($\le 2p$). A modification is going to consist of three things (the version in which something was changed, the field that got changed, and the value that got changed.)
		\item To read a field at version $v$, look through all the modifications with versions $\le v$ (there are constant many, $\le 2p$).
		\item Changing a field is hard, since there may not be any room in the mod structure.To modify $\text{node.field} = x$, we first check if there is any space in the mode structure. 
		\begin{itemize}
			\item If the node is not full, add a $mod$ $(now, field, x)$.
			\item If the node is full, 
		\begin{itemize}
			\item Make a new $node'$ with all the modifications applied. It will still have back pointers. Initially, the $mod$s will be empty.
			\item Change back pointers to $node \to node'$.
			\item Recursively update pointers.
		\end{itemize}
	\item Add back pointer from $x$ to $node$.
	\end{itemize}
\end{itemize}
The potential $\phi = c \cdot \sum \# \text{of mods in nodes in latest version} \Longrightarrow$ amortized cost $\le c + c - 2cp + p \text{ recursions}\le 2c$. 
\end{proof}
\end{document}

